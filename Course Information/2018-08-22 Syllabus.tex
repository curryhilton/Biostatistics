\documentclass[11pt, a4paper]{article}
%\usepackage{geometry}
\usepackage[inner=1.5cm,outer=1.5cm,top=2.5cm,bottom=2.5cm]{geometry}
\pagestyle{empty}
\usepackage{graphicx}
\usepackage{fancyhdr, lastpage, bbding, pmboxdraw}
\usepackage[usenames,dvipsnames]{color}
\definecolor{darkblue}{rgb}{0,0,.6}
\definecolor{darkred}{rgb}{.7,0,0}
\definecolor{darkgreen}{rgb}{0,.6,0}
\definecolor{red}{rgb}{.98,0,0}
\usepackage[colorlinks,pagebackref,pdfusetitle,urlcolor=darkblue,citecolor=darkblue,linkcolor=darkred,bookmarksnumbered,plainpages=false]{hyperref}
\renewcommand{\thefootnote}{\fnsymbol{footnote}}

\pagestyle{fancyplain}
\fancyhf{}
\lhead{ \fancyplain{}{Course Name} }
%\chead{ \fancyplain{}{} }
\rhead{ \fancyplain{}{\today} }
%\rfoot{\fancyplain{}{page \thepage\ of \pageref{LastPage}}}
\fancyfoot[RO, LE] {page \thepage\ of \pageref{LastPage} }
\thispagestyle{plain}

%%%%%%%%%%%% LISTING %%%
\usepackage{listings}
\usepackage{caption}
\DeclareCaptionFont{white}{\color{white}}
\DeclareCaptionFormat{listing}{\colorbox{gray}{\parbox{\textwidth}{#1#2#3}}}
\captionsetup[lstlisting]{format=listing,labelfont=white,textfont=white}
\usepackage{verbatim} % used to display code
\usepackage{fancyvrb}
\usepackage{acronym}
\usepackage{amsthm}
\VerbatimFootnotes % Required, otherwise verbatim does not work in footnotes!



\definecolor{OliveGreen}{cmyk}{0.64,0,0.95,0.40}
\definecolor{CadetBlue}{cmyk}{0.62,0.57,0.23,0}
\definecolor{lightlightgray}{gray}{0.93}



\lstset{
%language=bash,                          % Code langugage
basicstyle=\ttfamily,                   % Code font, Examples: \footnotesize, \ttfamily
keywordstyle=\color{OliveGreen},        % Keywords font ('*' = uppercase)
commentstyle=\color{gray},              % Comments font
numbers=left,                           % Line nums position
numberstyle=\tiny,                      % Line-numbers fonts
stepnumber=1,                           % Step between two line-numbers
numbersep=5pt,                          % How far are line-numbers from code
backgroundcolor=\color{lightlightgray}, % Choose background color
frame=none,                             % A frame around the code
tabsize=2,                              % Default tab size
captionpos=t,                           % Caption-position = bottom
breaklines=true,                        % Automatic line breaking?
breakatwhitespace=false,                % Automatic breaks only at whitespace?
showspaces=false,                       % Dont make spaces visible
showtabs=false,                         % Dont make tabls visible
columns=flexible,                       % Column format
morekeywords={__global__, __device__},  % CUDA specific keywords
}

%%%%%%%%%%%%%%%%%%%%%%%%%%%%%%%%%%%%
\begin{document}
\begin{center}
{\Large \textsc{Introduction to Biostatistics - STA 102}}
\end{center}
\begin{center}
Fall 2018
\end{center}
%\date{August 22, 2018}

\noindent\textbf{Course Goals and Objectives:}  

\vspace{5mm}

\noindent{This course introduces students to the discipline of statistics as a science of understanding
and analyzing data. Throughout the semester, students will learn how to effectively make use
of data in the face of uncertainty: how to collect data, how to analyze data, and how to use
data to make inferences and conclusions about real world phenomena. The course goals are as follows:}

\begin{enumerate}
\item Recognize the importance of data collection, identify limitations in data collection
methods, and determine how they affect the scope of inference.
\item Use statistical software to summarize data numerically and visually, and to perform data
analysis.
\item Have a conceptual understanding of the unified nature of statistical inference.
\item Apply estimation and testing methods to analyze single variables or the relationship
between two variables in order to understand natural phenomena and make data-based
decisions.
\item Model numerical response variables using a single or multiple explanatory variables.
\item Interpret results correctly, effectively, and in context without relying on statistical jargon.
\item Critique data-based claims and evaluate data-based decisions.
\item Complete a research project demonstrating mastery of statistical data analysis from
exploratory analysis to inference to modeling. 

\end{enumerate}



\begin{center}
\rule{6in}{0.4pt}
\begin{minipage}[t]{.75\textwidth}
\begin{tabular}{llcccll}
\textbf{Instructor:} & Curry W. Hilton & & &  & \textbf{Phone:} & 984.999.5481 \\
\textbf{Email:} &  \href{mailto:curry.hilton@duke.edu}{curry.hilton@duke.edu} & & & & \textbf{Office:} & 122 A Old Chem.
%TTH 3:05 P - 4:20 P (001), M 3:05 P - 4:20 P (01L), M 4:40 P - 5:55 P (02L)
\end{tabular}
\end{minipage}
\rule{6in}{0.4pt}
\end{center}
\vspace{.5cm}
\setlength{\unitlength}{1in}
\renewcommand{\arraystretch}{2}

\noindent\textbf{Course Pages:} \begin{enumerate}
\item \url{http://yourWebPage1.com/teaching}
\item \url{http://yourWebPage2.com/teaching}
\end{enumerate}

\vskip.15in
\noindent\textbf{Office Hours:} After class, or by appointment, or post your questions in the forum provided for this purpose on AeLP.

\vskip.15in
\noindent\textbf{Main References:} %\footnotemark
This is a  restricted list of various interesting and useful books that will be touched during the course. You need to consult them occasionally.
\begin{itemize}
\item Christopher M. Bishop, {\textit{Pattern Recognition and Machine Learning}}, Springer, 2006.
\item Peter J. Carrington, John Scott, and Stanley Wasserman, {\textit{Models and Methods in Social Network Analysis}}, Cambridge University Press, 2005.
\item Richard O. Duda, Peter E. Hart, and David G. Stork, {\textit{Pattern Classification}}, Wiley, 2nd ed., 2000.
\item Peter Flach, {\textit{Machine Learning: The Art and Science of Algorithms that Make Sense of Data}}, Cambridge University Press, 2012.

\end{itemize} 

% \footnotetext{Downloadable ebook versions are available on AeLP.}

\vskip.15in
\noindent\textbf{Objectives:}  This course is  primarily designed for graduate students ... 

\vskip.15in
\noindent\textbf{Prerequisites:}
An undergraduate-level understanding of probability, statistics, graph theory, algorithms, and linear algebra is assumed. 


\vspace*{.15in}

\noindent \textbf{Tentative Course Outline:}
\begin{center} 
\begin{minipage}{5in}
\begin{flushleft}
%Chapter 1 \dotfill ~$\approx$ 3 days \\
{\color{darkgreen}{\Rectangle}} ~A little of probability theory and graph theory	
\end{flushleft}
\end{minipage}
\end{center}

\vspace*{.15in}
\noindent\textbf{Grading Policy:} Homework and quizzes (30\%),  Midterm 1 (20\%), Midterm 2 (20\%), Final (30\%). %Four Projects (40\% = 4 * 10\%)

\vskip.15in
\noindent\textbf{Important Dates:}
\begin{center} \begin{minipage}{3.8in}
\begin{flushleft}
Midterm \#1      \dotfill ~\={A}b\={a}n 16, 1393  \\
Midterm \#2      \dotfill ~\={A}zar 21, 1393  \\
%Project Deadline \dotfill ~Month Day \\
Final Exam       \dotfill ~Dey 18, 1393  \\
\end{flushleft}
\end{minipage}
\end{center}

\vskip.15in
\noindent\textbf{Course Policy:}  
\begin{itemize}
\item Please sign up for AeLP. I will confirm your enrollment for the course, then you will be able to see the course page.

\end{itemize}

\vskip.15in
\noindent\textbf{Class Policy:}  
\begin{itemize}
\item Regular attendance is essential and expected.
\end{itemize}

\vskip.15in
\noindent\textbf{Academic Honesty:}   Lack of knowledge of the academic honesty policy is not a reasonable explanation for a violation.


%%%%%% THE END 
\end{document} 
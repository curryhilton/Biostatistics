\documentclass[11pt, a4paper]{article}
%\usepackage{geometry}
\usepackage[inner=1.5cm,outer=1.5cm,top=2.5cm,bottom=2.5cm]{geometry}
\pagestyle{empty}
\usepackage{graphicx}
\usepackage{fancyhdr, lastpage, bbding, pmboxdraw}
\usepackage[usenames,dvipsnames]{color}
\definecolor{darkblue}{rgb}{0,0,.6}
\definecolor{darkred}{rgb}{.7,0,0}
\definecolor{darkgreen}{rgb}{0,.6,0}
\definecolor{red}{rgb}{.98,0,0}
\usepackage[colorlinks,pagebackref,pdfusetitle,urlcolor=darkblue,citecolor=darkblue,linkcolor=darkblue,bookmarksnumbered,plainpages=false]{hyperref}
\renewcommand{\thefootnote}{\fnsymbol{footnote}}
\usepackage{flafter} 
\usepackage{booktabs}

\pagestyle{fancyplain}
\fancyhf{}
\lhead{ \fancyplain{}{Course Name} }
%\chead{ \fancyplain{}{} }
\rhead{ \fancyplain{}{\today} }
%\rfoot{\fancyplain{}{page \thepage\ of \pageref{LastPage}}}
\fancyfoot[RO, LE] {page \thepage\ of \pageref{LastPage} }
\thispagestyle{plain}

%%%%%%%%%%%% LISTING %%%
\usepackage{listings}
\usepackage{caption}
\DeclareCaptionFont{white}{\color{white}}
\DeclareCaptionFormat{listing}{\colorbox{gray}{\parbox{\textwidth}{#1#2#3}}}
\captionsetup[lstlisting]{format=listing,labelfont=white,textfont=white}
\usepackage{verbatim} % used to display code
\usepackage{fancyvrb}
\usepackage{acronym}
\usepackage{amsthm}
\VerbatimFootnotes % Required, otherwise verbatim does not work in footnotes!



\definecolor{OliveGreen}{cmyk}{0.64,0,0.95,0.40}
\definecolor{CadetBlue}{cmyk}{0.62,0.57,0.23,0}
\definecolor{lightlightgray}{gray}{0.93}



\lstset{
%language=bash,                          % Code langugage
basicstyle=\ttfamily,                   % Code font, Examples: \footnotesize, \ttfamily
keywordstyle=\color{OliveGreen},        % Keywords font ('*' = uppercase)
commentstyle=\color{gray},              % Comments font
numbers=left,                           % Line nums position
numberstyle=\tiny,                      % Line-numbers fonts
stepnumber=1,                           % Step between two line-numbers
numbersep=5pt,                          % How far are line-numbers from code
backgroundcolor=\color{lightlightgray}, % Choose background color
frame=none,                             % A frame around the code
tabsize=2,                              % Default tab size
captionpos=t,                           % Caption-position = bottom
breaklines=true,                        % Automatic line breaking?
breakatwhitespace=false,                % Automatic breaks only at whitespace?
showspaces=false,                       % Dont make spaces visible
showtabs=false,                         % Dont make tabls visible
columns=flexible,                       % Column format
morekeywords={__global__, __device__},  % CUDA specific keywords
}

%%%%%%%%%%%%%%%%%%%%%%%%%%%%%%%%%%%%
\begin{document}
\begin{center}
{\Large \textsc{Introduction to Biostatistics - STA 102}}
\end{center}
\begin{center}
Fall 2018
\end{center}
%\date{August 22, 2018}

\noindent\textbf{Course Goals and Objectives:}  

\vspace{2mm}

\noindent{This course introduces students to the discipline of statistics as a science of understanding
and analyzing data. Throughout the semester, students will learn how to effectively make use
of data in the face of uncertainty: how to collect data, how to analyze data, and how to use
data to make inferences and conclusions about real world phenomena. The course goals are as follows:}

\begin{enumerate}
\item Recognize the importance of data collection, identify limitations in data collection
methods, and determine how they affect the scope of inference.
\item Use statistical software to summarize data numerically and visually, and to perform data
analysis.
\item Have a conceptual understanding of the unified nature of statistical inference.
\item Apply estimation and testing methods to analyze single variables or the relationship
between two variables in order to understand natural phenomena and make data-based
decisions.
\item Model numerical response variables using a single or multiple explanatory variables.
\item Interpret results correctly, effectively, and in context without relying on statistical jargon.
\item Critique data-based claims and evaluate data-based decisions.
\item Complete a research project demonstrating mastery of statistical data analysis from
exploratory analysis to inference to modeling. 

\end{enumerate}

\noindent\textbf{Course Information:}

\begin{table}[h]
\centering
\begin{tabular}{l|l|l|l}
\toprule
                                & \textbf{Days}      & \textbf{Times} & \textbf{Location} \\ \midrule
Lecture                & TTH                   & 3:05 PM - 4:20 PM                   & Sociology - Psychology 126             \\ 
Lab 01                 & M                     & 3:05 PM - 4:20 PM                   & Social Sciences 119                    \\ 
Lab 02                 & M                     & 4:40 PM - 5:55 PM                   & Sociology - Psychology 129             \\ 
Office Hours (Curry)   & T                     & 4:30 PM - 6:30 PM                   & Old Chemistry 122 A                    \\ 
Office Hours (Patrick) & M &      12:00 Noon - 2:00 PM                               &     Old Chemistry 203 B                                  \\ 
Office Hours (TBD)  &  &                                     &                                        \\ \bottomrule
\end{tabular}
\end{table}


\noindent\textbf{Contact Information:}

\begin{table}[h]
\centering
\begin{tabular}{l|l|l|l}
\toprule
                         & \textbf{Office}& \textbf{Email} & \textbf{Phone} \\  \midrule
Curry W. Hilton & Old Chemistry 122 A                  & curry.hilton@duke.edu               & 984.999.5481                        \\ 
Patrick LeBlanc  (TA) & Old Chemistry 203 B             & patrick.leblanc@duke.edu            &                                     \\
TBD (TA) & Old Chemistry 203 B    & &                         \\ \bottomrule
\end{tabular}
\end{table}


\noindent\textbf{Course Resources:}

\begin{itemize}

\item OpenIntro Biostatistics: Introductory Statistics for the Life and Biomedical Sciences (Provided via Sakai)

\end{itemize}

\pagebreak


\noindent\textbf{Grading:}

\vspace{2mm}

\noindent{Your final grade will be comprised of the following.}

\begin{table}[h]
\centering
\begin{tabular}{|l|c|}
\toprule
                         & \textbf{Contribution \%}\\  \midrule
Labs  & 15 \%              \\
Exam 1 & 20 \%                       \\ 
Exam 2 & 20 \%                       \\ 
Final Exam & 25 \% \\
Project & 20 \% \\ \bottomrule
\end{tabular}
\end{table}

\noindent{The exact ranges of letter grades and +/- assignments will be determined after the final exam.  However, if you make 90\% or higher you are guaranteed at least a A-, 80\% or higher you are guaranteed at least B-, etc. } 

\vspace{5mm}

\noindent\textbf{Students with Disabilities:}

\vspace{2mm}

\noindent{Students with disabilities who believe they may need accommodations in this class are encouraged to contact the Student Disability Access Office (http://www.access.duke.edu/students/requesting/index.php) at (919) 668-1267 as soon as possible to better ensure that such accommodations can be made.}

\vspace{5mm}

\noindent\textbf{Academic Integrity:}

\vspace{2mm}

\noindent{Duke University is a community dedicated to scholarship, leadership, and service and to the
principles of honesty, fairness, respect, and accountability. Citizens of this community
commit to reflect upon and uphold these principles in all academic and non-academic
endeavors, and to protect and promote a culture of integrity. Cheating on exams and quizzes,
plagiarism on homework assignments and project, lying about an illness or absence and
other forms of academic dishonesty are a breach of trust with classmates and faculty, violate
the Duke Community Standard (https://studentaffairs.duke.edu/conduct/about-us/dukecommunity-standard),
and will not be tolerated. Such incidences will result in a 0 grade for all
parties involved as well as being reported to the Office of Student Conduct
(http://www.studentaffairs.duke.edu/conduct). Additionally, there may be penalties to your
final class grade. Please review the Duke?s Academic Dishonesty policies
(https://studentaffairs.duke.edu/conduct/z-policies/academic-dishonesty).}

\vspace{5mm}

\noindent\textbf{Excused Absences:}

\vspace{2mm}

\noindent{Students who miss graded work due to a scheduled varsity trip, religious holiday or shortterm
illness should fill out an online NOVAP (https://trinity.duke.edu/undergraduate/academicpolicies/athletic-varsity-participation),
religious observance notification (https://trinity.duke.edu/undergraduate/academic-policies\\/religious-holidays), or short-term
illness notification (http://trinity.duke.edu/academic-requirements?p=\\policy-short-termillness-notification)
form respectively. If you cannot complete an assignment on the due date due to a short-term illness, you have
until noon the following day to complete it at no penalty. Then the regular late work policy
will kick in.  If you are faced with a personal or family emergency or a long-range or chronic health
condition that interferes with your ability to attend or complete classes, you should contact
your academic dean?s office. See more information on policies surrounding these conditions
here (https://trinity.duke.edu/undergraduate/academic-policies/personal-emergencies). Your
academic dean can also provide more information.}

\vspace{5mm}

\noindent\textbf{Course Policies:}

\begin{itemize}

\item Late work policy for labs reports:
 
 \begin{itemize}
 
  \item Next day: lose 30\% of total possible points
  \item Later than next day: lose all points
 
 \end{itemize}
 
 \item  Late work policy for the project: 10\% off for each day late.
 \item There will be no make-ups for labs, 
project, or exams. If the midterm exam must be missed, absence must be officially
excused \textbf{in advance}, in which case the missing exam score will be imputed using the final
exam score. This policy only applies to the midterm. All other missed assessments will
receive a grade of 0. The final exam must be taken at the stated time. You must take the
final exam and turn in the project in order to pass this course.
\item Regrade requests must be made \textbf{within one week} of when the assignment is returned,
and must be submitted in writing. These will be honored if points were tallied incorrectly, or
if you feel your answer is correct but it was marked wrong. No regrade will be made to alter
the number of points deducted for a mistake. There will be no grade changes after the final
exam.
\item Use of disallowed materials (textbook, class notes, web references, any form of
communication with classmates or other persons, etc.) during exams will not be tolerated.
This will result in a 0 on the exam for all students involved, possible failure of the course,
and will be reported to the Office of Student Conduct
(http://www.studentaffairs.duke.edu/conduct). If you have any questions about whether something is or is not allowed, ask me beforehand.

\end{itemize}

%%%%%% THE END 
\end{document} 